\section{Kosten-Nutzen-Betrachtung}
Die Gesamtprojektkosten, von \EUR{3.640,00} setzen sich wie folgt
zusammen:
\begin{table}[htb]
\centering
\begin{tabular}{l l r r r}
\toprule
Tätigkeit & Posten & Kosten/Std. & Zeit & Kosten \\
\midrule
Planung & Gehalt -- Project owner & \EUR{70,00}& 5,0h & \EUR{350,00}\\
Umsetzung & Gehalt -- Auszubildender & \EUR{45,00}& 70,0h & \EUR{3150,00}\\
Abnahme & Gehalt -- Project owner & \EUR{70,00}& 2,0h & \EUR{140,00}\\
\cmidrule{5-5}
&&&& \EUR{3640,00}\\
\bottomrule
\end{tabular}
\caption{Aufwand Kosten}
\end{table}

\noindent Da die Planung und Abnahme seitens der Auszubildenden zum
Projektumfang von 70 Stunden gehören, sind die Kosten in dem Posten
Umsetzung zusammengefasst.
\\
Die Kosten für den Betrieb der entwickelten Software werden hier nicht
berücksichtigt, da die Anwendung auf interner Serverarchitektur
ausgeführt wurde.

Für den Nutzen der Software lässt sich derzeit nur schwer ein Geldwert
angeben. Er besteht in der verbesserten Qualität der Open Spaces und
der Möglichkeit, das Projekt als Referenz heranzuziehen, um daraus
neue Projekte zu akquirieren. Die Software kann zusätzlich eingesetzt
werden um das moderne, agile und innovative Arbeitsumfeld von OPITZ
Consulting zu demonstrieren und damit bei der Einstellung von neuen
Mitarbeitern zu helfen.

%%% Local Variables:
%%% mode: latex
%%% TeX-master: "../Doku"
%%% End:
