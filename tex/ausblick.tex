\section{Ausblick}
Die Applikation ist in ihrem jetzigen Stand noch an vielen Stellen
erweiterbar und es gibt einige Ideen für die Weiterentwicklung.

\subsection{Native Mobile}
Der native Support für Mobile Geräte ist noch nicht gegeben. Eine
interessante Entwicklung in dieser Richtung ist die Ankündung von
\textbf{React Native}\footnote{\url{http://goo.gl/LXvXLw}}.
Hier könnten sich neue Wege eröffnen native Apps als Webapplikationen
zu entwicklen.

\subsection{Eventstore als Persistenzlayer}
Die Anwendung hält sich streng an das \gls{Event Sourcing} Pattern. Es
modelliert die Kommunikation zwischen den Systembestandteilen als Event
Streams. Für diese Art von Eventbasierter Kommunikation wurde eine spezielle
Datenbank\footnote{\url{https://geteventstore.com/}} entwickelt die
das Observieren dieser Eventstreams unterstützt. Dies könnte
Performance Steigerungen und neue Möglichkeiten für die Verbesserung
der Fehlertoleranz der Applikation bringen.

\subsection{Nutzung als Referenzprojekt}
Das entstandene Projekt kann als Grundlage dienen, um
funktionale Konzepte in einer realistischen Umgebung auszuprobieren.
PureScript ist wahrscheinlich zu radikal um eine breite Nutzerbasis
anzusprechen, jedoch setzen sich Typsicherheit und funktionale
Konzepte im Browser immer mehr durch. So wird beispielsweise Microsoft's
TypeScript die Sprache sein, in der das aktuell sehr weit verbreitete MVC
Framework AngularJS in der nächsten Version entwickelt wird\footnote{\url{http://t3n.de/news/angularjs-20-typescript-597814/}}.
Facebooks ``typed Javascript'' ist
Flow\footnote{\url{http://flowtype.org/}}, ein JavaScript Compiler, der
JavaScript um ein Typsystem erweitert. Um diesen Entwicklungen
frühzeitig entgegen zu treten ist die Beschäftigung mit PureScript
eine wertvolle Erfahrung für jeden Frontend-Entwickler.

%%% Local Variables:
%%% mode: latex
%%% TeX-master: "../Doku"
%%% End:
