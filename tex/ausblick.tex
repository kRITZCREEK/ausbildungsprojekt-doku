\section{Ausblick}
Die Applikation ist in ihrem jetzigen Stand noch an vielen Stellen
erweiterbar und es gibt einige Ideen für die Weiterentwicklung. Der
native Support für Mobile Geräte ist noch nicht gegeben. Eine
interessante Entwicklung in dieser Richtung ist die Ankündung von
\textbf{React Native}.
\footnote{\url{https://code.facebook.com/videos/786462671439502/react-js-conf-2015-keynote-introducing-react-native-/}}
Hier könnten sich neue Wege eröffnen native Apps als Webapplikationen
zu entwicklen. Das entstandene Projekt kann als
Spielwiese/Lernmaterial dienen um funktionale Konzepte in einer
realistischen Umgebung auszuprobieren. PureScript ist wahrscheinlich
zu radikal um eine breite Nutzerbasis anzusprechen. Typsicherheit und
funktionale Konzepte im Browser setzen sich jedoch immer mehr durch.
So wird beispielsweise Microsoft's TypeScript die Sprache sein in der
das aktuell wichtigste MVC Framework AngularJS entwickelt wird.
\footnote{\url{http://t3n.de/news/angularjs-20-typescript-597814/}}
Facebooks ``Typed Javascript'' ist \textbf{Flow} ein JavaScript
Compiler der JavaScript um ein Typsystem erweitert. Um
diesen Entwicklungen frühzeitig, progressiv entgegen zu treten ist die
Beschäftigung mit PureScript eine wertvolle Erfahrung für jeden
Frontend-Entwickler.

%%% Local Variables:
%%% mode: latex
%%% TeX-master: "../Doku"
%%% End:
