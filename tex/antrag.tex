\section{Projektantrag}
\section*{Projektbezeichnung/Auftrag/Teilauftrag}
Entwicklung einer mobilen Webanwendung zur Planung und Unterstützung
von Open Spaces.
\subsection*{Ausgangssituation}
Die als Auftraggeber auftretende OPITZ CONSULTING Deutschland GmbH ist
ein mittelständisches, wachsendes Unternehmen im Bereich der
IT-Beratung und Dienstleistung. Am Hauptsitz in Gummersbach befindet
sich neben der Holding Gesellschaft noch der Größte der insgesamt neun
Standorte.

\noindent Seit einigen Jahren werden bei Opitz regelmäßig Open Spaces durchgeführt.
Ein Open Space ist eine Veranstaltung, bei der mehrere Personen
zusammen kommen, Themen vorschlagen und diese in einer wie auch immer
gearteten Weise bearbeiten, besprechen oder erlernen. Ein Open Space
verteilt sich hierbei auf mehrere Räume und Timeslots, die zuvor
festgelegt werden müssen. Die Themen werden von den Teilnehmern selbst
angeboten und die Teilnehmer entscheiden mit ihrer Interessenbekundung
an einem oder mehreren Themen, wie sich das Programm gestaltet.
(siehe auch:\footnote{\url{http://en.wikipedia.org/wiki/Open_Space_Technology}})
Falls die Teilnehmer nach Ablauf der Zeit eines Themas weiteres
Interesse anmelden kann das Thema ad hoc verlängert und verlegt werden.

\noindent Derzeit verläuft die Planung dieser Open Spaces direkt auf
der Veranstaltung mithilfe von Whiteboards oder Flipcharts. Im
Nachhinein wird dann zumeist in einem CMS über die Veranstaltungen
berichtet und Bilder hochgeladen, wobei dies auf freiwilliger Basis
geschieht und zusätzlichen Aufwand bedeutet.

\noindent Dieses sehr statische Planungsvorgehen wird dem dynamischen
Charakter von Open Spaces nicht gerecht. Als verteilte Veranstaltung
mit ``Realtime''-Aspekt ist für eine sinnvolle Kommunikation ein hohes
Maß an Vernetzung unverzichtbar. Das dokumentieren der parallelen und
verteilten Vorgänge ist aus der Sicht einer Person aufwendig und das
Ergebnis lückenhaft. Eine mobile Webanwendung kann und soll hier
Abhilfe schaffen.
\subsection*{Zielsetzung}
Ziel des Projektes ist es, eine erweiterbare Full-Stack-Anwendung zu
entwickeln, welche die Durchführung von Open Spaces unterstützt. Der
Veranstaltungscharakter legt den Anwendungsfall nahe, eine Instanz
dieser Applikation bei Bedarf zu starten und nach der Durchführung der
Veranstaltung wieder zu beenden. Hieraus lässt sich die Cloud als
geeignete Plattform ableiten. Eine weitere Anforderung an die
Anwendung ist ein mobiler Client, der den Teilnehmern eine Sicht auf
den aktuellen Zeitplan und die Aufnahme und Verteilung von Bildern und
Notizen ermöglicht. Außerdem sollen über den mobilen Client in einem
Push-basierten Modell Notifikationen über Änderungen am Zeitplan
verteilt werden.
Ein weiteres Ziel ist die Evaluierung von funktionalen
Programmiertechniken im Bezug auf ihre Tauglichkeit für die Erstellung
von Webapplikationen.
\subsection*{Konsequenzen bei Nichtverwirklichung}
Open Spaces werden weiterhin am Whiteboard geplant und Änderungen am
Zeitplan erreichen  Teilnehmer nur dann, wenn diese am zentralen Board
nachgetragen werden und die Teilnehmer sich dort regelmäßig über
Änderungen informieren. Die Dokumentation des zeitlichen Ablaufs
gestaltet sich weiterhin als schwierig.
\section*{Projektumfeld/Rahmenbedingungen}
Um möglichst viele Plattformen (Windows, MacOS, Linux, iOS, Android)
und Geräte (Dekstop, Mobile Endgeräte) ansprechen zu können, soll die
Entwicklung der Anwendung auf Webtechnologien basieren. Als
Versionierungsmodell soll ein Continuous Integration System aufgesetzt
werden.
\section*{Projektplanung}
\subsubsection*{Zeitplanung}
\begin{tabular}{l l r}
1. & Planung & 2h \\
2. & Entwurf & 5h \\
\toprule 3. & Implementierung & \\
3.1 & Browser Frontend & 15h \\
3.2 & Backend & 20h \\
3.3 & Mobile Webapp & 5h \\
\toprule 4. & Abschließender Test & 5h \\
5. & Übergabe & 3h \\
6. & Dokumentation & 15h \\
\toprule $\Sigma$ & Summe & 70h
\end{tabular}

\section*{Anzufertigende Dokumente}
\begin{itemize}
\item Projektdokumentation
\item Selbstständigkeitserklärung
\item Screenshots der Anwendung
\item Quellcode
\end{itemize}
