\newglossaryentry{ghc}
{
  name=Glasgow Haskell Compiler,
  description={ist der meistgenutzte und fortschrittlichste Haskell Compiler.
    Er ist Open-Source und wird seit seiner ersten lauffähigen Version im Juni 1989 weiterentwickelt\\
    \url{https://www.haskell.org/ghc/}
  }
}
\newglossaryentry{cabal}
{
  name=cabal-install,
  description={ist das de-facto Standard Build und Packaging Tool für Haskell Programme und Bibliotheken\\
    \url{https://www.haskell.org/cabal/}
  }
}
\newglossaryentry{postgresql}
{
  name=PostgreSQL,
  description={ist eine relationale Datenbank die Open-Source entwickelt wird\\
    \url{http://www.postgresql.org/}
  }
}
\newglossaryentry{psc}
{
  name=PureScript Compiler,
  description={ist der PureScript Compiler, der PureScript Source Files in JavaScript übersetzt.\\
    \url{http://www.purescript.org/}
  }
}
\newglossaryentry{react-tools}
{
  name=React Tools,
  description={ist eine Sammlung von Tooling welche bei der Entwicklung von React Komponenten helfen\\
    \url{http://facebook.github.io/react/}
  }
}
\newglossaryentry{pulp}
{
  name=pulp,
  description={ist ein von Bodil Stokke entwickeltes Build-System für PureScript\\
    \url{https://github.com/bodil/pulp}
  }
}
\newglossaryentry{bower}
{
  name=bower,
  description={ist ein Tool für Dependency-Management von Javascript Bibliotheken. Auch die PureScript Community
    verwendet es um Bibliotheken zu verbreiten und verwalten\\
    \url{http://bower.io/}
  }
}
\newglossaryentry{CMS}
{
  name=CMS,
  description={steht für Content-Management-System und beschreibt eine Software zur gemeinschaftlichen
    Erstellung, Bearbeitung und Organisation von Inhalten. In diesem Fall handelt es sich um die Software
    Confluence von Atlassian
  }
}
\newglossaryentry{Event Sourcing}
{
  name=Event Sourcing,
  description={beschreibt ein Vorgehen bei dem alle Zustandveränderungen
    innerhalb eines Systems auf Events abgebildet werden. Dies erlaubt es Vorgänge
    erneut abzuspielen oder die Vergangenheit zu verändern
  }
}
\newglossaryentry{ssh}
{
  name=Secure Shell (ssh),
  description={bezeichnet sowohl ein Netzwerkprotokoll als auch entsprechende Programme, mit deren Hilfe man auf eine sichere Art und Weise eine verschlüsselte Netzwerkverbindung mit einem entfernten Gerät herstellen kann
  }
}
\newglossaryentry{Reactive Programming}
{
  name=Reactive Programming,
  description={beschreibt ein Paradigma, welches sich um Datenflüsse und das Propagieren von Änderungen durch diese dreht.
  Es hat in neuester Zeit an Beliebtheit gewonnen und wird als wichtiger Ansatz gesehen um robuste, skalierbare Systeme zu entwickeln\\
  \url{http://www.reactivemanifesto.org/}
  }
}
\newglossaryentry{React}
{
  name=React,
  description={ist ein von Facebook entwickeltes UI-Framework das ein funktionalen Ansatz für das Erstellen von Oberflächen empfiehlt\\
  \url{http://facebook.github.io/react/}
  }
}
\newglossaryentry{Yesod Framework}
{
  name=Yesod Framework,
  description={ist ein Open-Source Web-Framework, welches in Haskell geschrieben ist und es sich zum Ziel macht Haskells Typsystem zu nutzen,
  um Laufzeitfehler in Compilezeitfehler umzuwandeln.\\
  \url{http://www.yesodweb.com/}
  }
}
\newglossaryentry{PureScript}
{
  name=PureScript,
  description={PureScript ist eine kleine stark getypte Programmiersprache die zu Javascript kompiliert wird.\\
  \url{http://www.purescript.org/}
  }
}

%%% Local Variables:
%%% mode: plain-tex
%%% TeX-master: "../Doku"
%%% End:
