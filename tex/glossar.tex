\newglossaryentry{ghc}
{
  name=Glasgow Haskell Compiler,
  description={ist der meistgenutzte und fortschrittlichste Haskell Compiler.
    Er ist Open-Source und wird seit seiner ersten lauffähigen Version im Juni 1989 weiterentwickelt\\
    \url{https://www.haskell.org/ghc/}
  }
}
\newglossaryentry{cabal}
{
  name=cabal-install,
  description={ist das de-facto Standard Build und Packaging Tool für Haskell Programme und Bibliotheken\\
    \url{https://www.haskell.org/cabal/}
  }
}
\newglossaryentry{postgresql}
{
  name=PostgreSQL,
  description={ist eine relationale Datenbank die Open-Source entwickelt wird\\
    \url{http://www.postgresql.org/}
  }
}
\newglossaryentry{psc}
{
  name=PureScript Compiler,
  description={ist der PureScript Compiler, der PureScript Source Files in JavaScript übersetzt.\\
    \url{http://www.purescript.org/}
  }
}
\newglossaryentry{react-tools}
{
  name=React Tools,
  description={ist eine Sammlung von Tooling welche bei der Entwicklung von React Komponenten helfen\\
    \url{http://facebook.github.io/react/}
  }
}
\newglossaryentry{pulp}
{
  name=pulp,
  description={ist ein von Bodil Stokke entwickeltes Build-System für PureScript\\
    \url{https://github.com/bodil/pulp}
  }
}
\newglossaryentry{bower}
{
  name=bower,
  description={ist ein Tool für Dependency-Management von Javascript Bibliotheken. Auch die PureScript Community
    verwendet es um Bibliotheken zu verbreiten und verwalten\\
    \url{http://bower.io/}
  }
}
