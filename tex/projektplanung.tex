\section{Projektplanung}
\subsection{IST-Analyse}
Auf die aktuelle Lage wurde bereits ausreichend in der Beschreibung
der Ausgangssituation eingegangen. Darum sollen hier nur noch einmal
die wichtigsten Punkte aufgelistet werden.
\begin{itemize}
\item Es gibt eine vorhandene Lösung für das Zusammentragen von Themen
  im Voraus
\item Fehlende Möglichkeit während des Open Space's Änderungen am
  Zeitplan vorzunehmen
\item Fehlende Möglichkeit diese Änderungen in Echtzeit an die
  Teilnehmer weiter zu reichen
\item Fehlende Möglichkeit für das Festhalten der durchgeführten
  Diskussionen, Workshops, etc\ldots
\end{itemize}
\subsection{SOLL-Konzept}
Gewünscht ist eine Möglichkeit die zuvor gesammelten Themenvorschläge
auf Räume und Zeitslots zu verteilen. Weiterhin soll es möglich sein
neue Themen hinzuzufügen. Der Zeitplan muss sich in Echtzeit anpassen
lassen und die vorgenommenen Änderungen sollen alle Teilnehmer des
Open Spaces erreichen. Als zusätzliche Anforderung soll die gewünschte
Applikation in einem funktionalen Stil entwickelt werden um spätere
Änderungswünsche leichter umsetzen zu können.
\subsection{Zielsetzung}
Primärziel ist es eine Webapplikation zur Unterstützung von Open
Spaces zu entwickeln. Weitere Ziele sind eine mobile Lösung um den
Zeitplan zu betrachten, sowie die Tauglichkeit von funktionalen
Techniken für die Webentwicklung zu evaluieren.
\subsection{Infrastrukturelle Rahmenbedingungen}
Für die Applikation wird ein Server im firmeninternen Netz zur
Verfügung gestellt. Die Administration dieses Servers geschieht über
\texttt{ssh} aus dem firmeninternen Netz. Der Server wird zeitlich
begrenzt für die Dauer des Open Spaces bereitgestellt.
\subsubsection{Entwicklungsumgebung}
Entwickelt wurde auf einem Ubuntu 14.04 Linux System. Als
Entwicklungsumgebung diente Emacs mit Plugins für den jeweiligen
Einsatzfall. Die Applikation wurde mit Chrome und Firefox auf dem
angesprochenen Linux System getestet. Während der Entwicklung lief der
Server sowie die Datenbank lokal.
\subsubsection*{Verwendetes Tooling am Backend:}
\begin{itemize}
\item \gls{ghc} in Version \texttt{7.8.3}
\item \gls{cabal} in Version \texttt{1.20.0.0}
\item \gls{postgresql} in Version \texttt{9.3.6}
\end{itemize}

\subsubsection*{Verwendetes Tooling am Frontend:}
\begin{itemize}
\item \gls{psc} in Version \texttt{0.6.8}
\item \gls{react-tools} in Version \texttt{0.12.2}
\item \gls{pulp} in Version \texttt{2.0.2}
\item \gls{bower} in Vesion \texttt{1.3.5}
\end{itemize}
Es wurde außerdem ein Buildserver in einer Virtuellen Maschine
aufgesetzt, der die spätere Laufzeitumgebung für den Server
nachbildet. Auf diesem Buildserver gebaute Programme können einfach
auf das Produktivsystem kopiert und dort benutzt werden.
\subsubsection{Laufzeitumgebung}
Das entwickelte Backend soll abschließend auf einem Debian Server
laufen. Die Webapplikation (das Frontend) soll sowohl im Browser auf
Desktop Clients, als auch in den Browsern der mobilen Clients
ausgeführt werden.
\subsection{Terminplanung}
Das Projekt ist zwischen dem 25.01.2015 und dem 17.04.2015 zu
realisieren und umfasst maximal 70 Arbeitsstunden. Am 17.03.2015
findet wieder ein Open Space bei Opitz statt. Hier soll das Tool
seinen ersten Einsatz haben und getestet werden.

%%% Local Variables:
%%% mode: latex
%%% TeX-master: "../Doku"
%%% End:
