\section{Ausgangssituation}
\subsection{Unternehmensbeschreibung}
Das IT-Beratungshaus OPITZ CONSULTING Deutschland GmbH ist ein
wachsendes, mittelständisches Unternehmen im Bereich der IT-Beratung
und -Dienstleistung. Am Hauptsitz in Gummersbach befinden sich die
Holding-Gesellschaft sowie der Größte der insgesamt neun Standorte.
Das Unternehmen ist seit seiner Gründung im Jahr 1990 auf derzeit etwa
450 Mitarbeiter gewachsen. Die 9 Standorte setzen sich aus sieben
Standorten in Deutschland, sowie zwei Standorten in Polen zusammen.
\subsection{Projekthintergrund}
Seit einigen Jahren werden bei OPITZ-CONSULTING regelmäßig Open Spaces
durchgeführt. Ein Open Space ist eine Veranstaltung, bei der mehrere
Personen zusammen kommen, Themen vorschlagen und diese in einer wie
auch immer gearteten Weise bearbeiten, besprechen oder erlernen. Ein
Open Space verteilt sich hierbei auf mehrere Räume und Timeslots, die
zuvor festgelegt werden müssen. Die Themen werden von den Teilnehmern
selbst angeboten und die Teilnehmer entscheiden mit ihrer
Interessenbekundung an einem oder mehreren Themen, wie sich das
Programm gestaltet. (siehe auch:\footnote{\url{http://en.wikipedia.org/wiki/Open_Space_Technology}})
Falls die Teilnehmer nach Ablauf der Zeit eines Themas weiteres
Interesse anmelden kann das Thema ad hoc verlängert und verlegt
werden.
\noindent Derzeit verläuft die Planung dieser Open Spaces direkt auf
der Veranstaltung mithilfe von Whiteboards oder Flipcharts. Im
Nachhinein wird dann zumeist in einem \gls{CMS} über die Veranstaltungen
berichtet und Bilder hochgeladen, wobei dies auf freiwilliger Basis
geschieht und zusätzlichen Aufwand bedeutet.

\noindent Dieses sehr statische Planungsvorgehen wird dem dynamischen
Charakter von Open Spaces nicht gerecht. Als verteilte Veranstaltung
mit ``Realtime''-Aspekt ist für eine sinnvolle Kommunikation ein hohes
Maß an Vernetzung unverzichtbar. Das dokumentieren der parallelen und
verteilten Vorgänge ist aus der Sicht einer Person aufwendig und das
Ergebnis lückenhaft. Eine mobile Webanwendung kann und soll hier
Abhilfe schaffen.

%%% Local Variables:
%%% mode: latex
%%% TeX-master: "../Doku"
%%% End:
